\documentclass{article}
\usepackage{german}
\begin{document}
	\section{Projektbeschreibung}
Ziel dieses Projekts ist es ein eingebettetes Betriebssystem f"ur einen Raspberry Pi zu erstellen. Dieses Raspberry Pi ist bereits auf einer Modelleisenbahn montiert und soll eine Kontrolle des Zuges und die R"uckgabe von Daten erm"oglichen.
Um eventuelle Ver"anderung am Kernel zu vereinfachen, soll ein Deploykonzept entwickeln, welches die "Anderungen "uber das Netzwerk erm"oglicht. Dies soll den mechanischen Zugriff auf die SD-Karte abl"osen.
Um ein ma"sgefertigtes Linux Kernel zu erschaffen k"onnen wir das Open-Source Werkzeug Yocto benutzen. Dieses erm"oglicht es ein Linux Betriebssystem f"ur ein eingebettetes System zu gestallten. Es erm"oglicht die Architektur des CPU's selbst zu gestalten, indem man Komponenten entfernen und hinzuf"ugen kann.
Eine bereits funktionsf"ahige Anwendungssoftware die mit IBM-Rhapsody entwickelt wurde soll in den Prozess integriert werden.
\end{document}