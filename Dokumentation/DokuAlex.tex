\documentclass[a4paper,10pt] {article}
\begin {document}
	\section{Netzwerkboot}	
		
		Um das Projekt zu realisieren wurde das Team in Mehrere Gruppen aufgeteilt. Jonas und Alexander versuchten den Netzwerkboot umzusetzen.
		
		Beim Netzwerkboot war das Ziel, dass der Pi 3 der im Zug verbaut ist, beim Start nicht über die SD Karte oder ein anderes angeschlossenes Speichergerät Startet, sondern über ein Betriebssystem Image das auf einem DHCP/TFTP Server liegt.
		Der große Vorteile hierbei ist, dass es künftig keine Probleme mehr mit kaputten SD Karten gibt.
		
		In ersten Tests richteten wir eine Pi 3 VM auf dem PC ein. Diese diente als Server einen Pi 3 den wir zur Verfügung hatten richteten wir als Client ein. 
		
		\subsection{Client Konfiguration}
		
			Der Client muss zuerst mithilfe einer SD Card gebootet werden. Mit dem Befehl
			
				echo program\_usb\_boot\_mode=1 | sudo tee -a /boot/config.txt
			
			schalten wir den USB boot mode ein. Um zu überprüfen ob der OTP Speicher im Pi nun richtig konfiguriert ist, starten wir den Pi neu und führen den Befehl
			
				vcgencmd otp\_dump | grep 17:
			
			aus. Die Ausgabe muss 0x3020000a sein. Falls die Ausgabe Korrekt ist, entfernen wir nun wieder den Eintrag in der config.txt. Der Client ist nun Konfiguriert. 
			
		\vfill
		
		\subsection{Server Konfiguration}
		
			Bei der Server Konfiguration erweitern wir das Filesystem auf die komplette SD Karte mit dem Konfigurationstool 
			
			sudo raspi-config
			
			Da der Client ein root filesystem benötigt zum booten, Erstellen wir ein Abbild des aktuellen Filesystem und hinterlegen es im Verzeichniss /nfs/client1.
			 
			sudo mkdir -p /nfs/client1
			sudo rsync -xa --progress --exclude /nfs / /nfs/client1
			
			Nun müssen die SSH Schlüssel erneuert werden
			
			cd /nfs/client1
			sudo mount --bind /dev dev
			sudo mount --bind /sys sys
			sudo mount --bind /proc proc
			sudo chroot .
			rm /etc/ssh/ssh\_host\_*
			dpkg-reconfigure openssh-server
			exit
			sudo umount dev
			sudo umount sys
			sudo umount proc
			
			wir benötigen nun noch die Netzwerkinformationen. Hierzu muss der Pi mit dem Netzwerk verbunden sein.
			
			ip route | grep default | awk '{print \$3}'
			ip -4 addr show dev eth0 | grep inet
			
			Die Ausgabe sollte so aussehen
			
			inet 192.168.1.101/24 brd 192.168.1.255 scope global eth0
			
			Server IP Adresse: 192.168.1.101
			Brodcast Adresse: 192.168.1.255
			eth0: Verbindung über LAN Kabel
			
			die IP Adresse vom DNS Server bekommen wir mit 
			
			cat /etc/resolv.conf
			
			der Pi selber benötigt noch eine Statische IP Adresse
			
			sudo nano /etc/network/Interfaces
			
			Hier die Zeile 
			
			iface eth0 inet manual 
			
			ersetzen mit 
			
			auto eth0
			iface eth0 inet static 
				address 192.168.1.2
				netmask 255.255.255.0
				gateway 192.168.1.1
				
			Die gateway Adresse ist die Adresse die wir aus resolv.conf ausgelesen haben
			
			den DHCP client Daemon müssen wir auch noch ausschalten
			
			sudo systemctl disable dhcpcd
			sudo systemctl enable Networking
			
			sudo reboot
			
			Mit dem Neustart wurden unsere Änderungen übernommen. Wir haben nun kein funktionierendes DNS mehr. Wir fügen nun unsere Server Konfiguration ein. nameserver entspricht der gateway Adresse.
			 
			echo "nameserver 192.168.1.1" | sudo tee /etc/resolv.conf
			
			die Datei darf durch dnsmasq nicht mehr verändert werden weshalb wir sie mit dem Befehl
			
			sudo chattr +i /etc/resolv.conf
			
			unveränderbar machen. Es ist nun zwingend erforderlich zusätzliche Software zu installieren 
			
			sudo apt-get update
			sudo apt-get install dnsmasq tcpdump
			
			Mit dem nächsten Befehl stoppen wir die DNS Namensauflösung mit dnsmasq 
			
			sudo rm /etc/resolvconf/update.d/dnsmasq
			sudo reboot
			
			Wir starten nun tcpdump um nach DHCP Paketen vom Client Raspberry Pi zu suchen
			
			sudo tcpdump -i eth0 port bootpc
						
		\vfill
		
		\subsection{Inbetriebnahme}
			
			Nachdem nun auch die Netzwerk config abgeschlossen ist schließen wir den Client Raspberry ans Netzwerk an und starten ihn. Nach 10 Sekunden sollten wir auf dem Client beobachten können, dass die LED anfängt zu leuchten. Auf dem Bildschirm sollten wir nun Pakete mit der Bezeichnung BOOTP/DHCP, Request bekommen.
			
			IP 0.0.0.0.bootpc > 255.255.255.255.bootps: BOOTP/DHCP, Request from b8:27:eb...
		
			Der Server empfängt nun Anfragen vom Client kann aber noch nicht Antworten. Hierfür müssen wir erst noch dnsmasq konfigurieren
			
			echo | sudo tee /etc/dnsmasq.conf
			sudo nano /etc/dnsmasq.conf
			
			Den gesamten Inhalt der Datei ersetzen durch
			
			port=0
			dhcp-range=192.168.1.255,proxy
			log-dhcp
			enable-tftp
			tftp-root=/tftpboot
			pxe-service=0,``Raspberry Pi Boot''
			
			dhcp-range = Brodcast Adresse
			
			Das Verzeichniss /tftboot müssen wir noch erstellen
			
			sudo mkdir /tftpboot
			sudo chmod 777 /tftpboot
			sudo systemctl enable dnsmasq.service
			sudo systemctl restart dnsmasq.service
			
			Wir überwachen jetzt das dnsmasq Protokoll mit 
			
			tail -F /var/log/daemon.log
			
			und suchen dort einen Einträgen die ungefähr so aussehen
			
			raspberrypi dnsmasq-tftp[1903]: file /tftpboot/bootcode.bin not found
			
			wir holen uns nun die Datei bootcode.bin und start.elf in unser /tftboot Verzeichniss indem wir das komplette boot Verzeichnis kopieren
			
			cp -r /boot/* /tftpboot
			sudo systemctl restart dnsmasq
			
			Für einen ersten Test verwendeten wir nun das am Anfang kopierte root filesystem in /nfs/client1 dieses sollte später durch unser in Yocto erstelltes root filesystem ersetzt werden
			
			sudo apt-get install nfs-kernel-server
			echo ``/nfs/client1 *(rw,sync,no\_subtree\_check,no\_root\_squash)'' | sudo tee -a /etc/exports
			sudo systemctl enable rpcbind
			sudo systemctl restart rpcbind
			sudo systemctl enable nfs-kernel-server
			sudo systemctl restart nfs-kernel-server
			
			/tftpboot/cmdline.txt muss nun noch angepasst werden
			
			root=/dev/nfs nfsroot=192.168.1.2:/nfs/client1 rw ip=dhcp rootwait elevator=deadline
			
			Die IP Adresse entspricht der vom Client. Abschließend müssen noch die SD Karten Einträge in fstab gelöscht werden
			
			sudo nano /nfs/client1/etc/fstab
			
			Hier die Einträge 
			
			/dev/mmcblkp1
			/dev/mmcblkp2
			
			löschen. Nun sollte der Netzwerkboot richtig konfiguriert sein.
			
		\vfill
		
		\subsection{rsync}
			Leider funktionierte der Netzwerkboot nicht. Wir haben im Internet noch diverse andere Anleitungen ausprobiert aber alle ohne Erfolg. In einem Versuch hatten wir das System soweit, dass sich der Client ein Paket vom Server holte, danach aber aufhörte. Als Alternative zum Netzwerkboot hinterlegten wir nun ein root filesystem auf dem Server im Labor im Verzeichnis /projects/ss18\_fhftrain/rfs. Dieses ist erreichbar über 
			
			Server:141.28.57.62 
			Port: 12345
			
			Das root filesystem ist mithilfe von Yocto erstellt worden und läuft auch auf dem Pi. Mithilfe von rsync übernimmt der Pi beim Einschalten alle Änderungen die im Verzeichnis /projects/ss18\_fhftrain/rfs gemacht wurden. 
			
			Als erstes muss rsync installiert werden
			
			sudo apt-get install rsync 
			
			Die Syntax des rsync Befehls sieht folgendermaßen aus
			
			rsync [OPTIONEN] QUELLE(N) ZIEL 
			
			Quelle: /projects/ss18\_fhftrain/rfs/
			Ziel: /
			
			Optionen:
			Es ist wichtig einige Pfade und Ordner zu excludieren. Diese enthalten unter anderem Hardware spezifische Daten die nicht überschrieben werden dürfen. Das Verzeichnis mit dem rsync Befehl müssen wir auch excludieren. 
			
			--exclude=/mnt 
			--exclude=/media 
			--exclude=/sys 
			--exclude=/tmp 
			--exclude=/var/log 
			--exclude=/etc 
			--exclude=/dev 
			--exclude=/proc 
			--exclude=/root 
			--exclude=/boot 
			--exclude=/lib/modules 
			--exclude=``lost+found''
			
			--exclude=/home/pi/Programm 
			--exclude=/home/pi/.ssh 
			
			Andere Optionen:
			
			-vvv
			gibt uns während des Synchronisierens Debuginfos aller ausgeführter Schritte aus 
			
			-av
			
			--delete
			vergleicht Quellverzeichnisse und Zielverzeichnisse und sorgt dafür, dass Dateien, die im Quellverzeichnis nicht (mehr) vorhanden sind, im Zielverzeichnis gelöscht werden			
		
		\vfill
		
\newpage

\end {document}


